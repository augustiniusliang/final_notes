\documentclass[UTF8]{ctexart}
\usepackage{amsmath, amssymb, amsthm}
\usepackage{geometry}
\geometry{a4paper, scale=0.8}
\usepackage{graphicx}
\title{高等微积分期末辅导}
\author{梁莫言, 清华大学未央书院.}
\newtheorem{problem}{问题}
\date{2025 年 1 月 5 日}

\begin{document}
\maketitle
\tableofcontents
\newpage

\section{不定积分的计算}
\subsection{分部积分法}
\begin{problem}
  计算以下不定积分:
  \[
    \int x^2 e^x \, \text{d}x, \quad \int \ln x \, \text{d}x, \quad \int e^x \sin x \, \text{d}x.
  \]
\end{problem}\vspace{90pt}
\begin{problem}
    计算以下不定积分:
    \[
        \int \cos^5 x \,\text{d}x, \quad \int \arcsin x \, \text{d}x, \quad \int \sec^3 x \, \text{d}x.
    \]
\end{problem}\vspace{90pt}
\begin{problem}
    推到下列积分的递推公式:
    \[
        I_n = \int \sin^n x \, \text{d}x, \quad J_n = \int x^n e^x \, \text{d}x, \quad K_n = \int \tan^n x \, \text{d}x.
    \]
\end{problem}\vspace{90pt}

\subsection{换元法. 三角换元, 双曲三角换元}
\begin{problem}
    计算以下不定积分:
    \[
        \int \frac{\text{d}x}{x^2 \sqrt{x^2 - 4}}, \quad \int \frac{\text{d}x}{\sqrt{1 - x^2}}, \quad \int \frac{\text{d}x}{\sqrt{x^2 + 4}}.
    \]
\end{problem}\vspace{90pt}
\begin{problem}
    计算以下不定积分:
    \[
        \int \frac{\text{d}x}{x \sqrt{x^2 - x + 1}}, \quad \int \frac{\text{d}x}{\sqrt{x^2 + x + 1}}, \quad \int \frac{\text{d}x}{x^2 \sqrt{4 - x^2}}.
    \]
\end{problem}\vspace{90pt}
\begin{problem}
    计算以下不定积分:
    \[
        \int \sinh^3 x \, \text{d}x, \quad \int \frac{\text{d}x}{\cosh x}, \quad \int \frac{\text{d}x}{\sinh x}.
    \]
\end{problem}\vspace{90pt}
\begin{problem}
    计算以下不定积分:
    \[
        \int \frac{\text{d}x}{\sqrt{x^2 + 1}}, \quad \int \frac{\text{d}x}{\sqrt{x^2 - 1}}, \quad \int \frac{\text{d}x}{\sqrt{1 - x^2}}.
    \]
\end{problem}\vspace{90pt}
\subsection{有理分式的化简与原函数}
\begin{problem}
    计算以下不定积分:
    \[
        \int \frac{2x^2 + 3x + 1}{(x - 1)(x^2 + 1)} \, \text{d}x, \quad \int \frac{x^3 + 1}{x^2 - x} \, \text{d}x.
    \]
\end{problem}\vspace{90pt}
\begin{problem}
    计算以下不定积分:
    \[
        \int \frac{\text{d}x}{x^3 - x^2}, \quad \int \frac{\text{d}x}{x^4 + 4}, \quad \int \frac{\text{d}x}{x^4 + x^2 + 1}.
    \]
\end{problem}\vspace{90pt}
\subsection{超越函数的原函数}
\begin{problem}
    计算以下不定积分:
    \[
        \int \sqrt{e^x-1} \, \text{d}x, \quad \int \frac{\text{d}x}{\sqrt{1 - e^x}}, \quad \int \frac{\text{d}x}{\sqrt{e^x + 1}}.
    \]
\end{problem}\vspace{90pt}

\begin{problem}
    计算以下不定积分:
    \[
        \int \frac{\text{d}x}{\sqrt{1 - x^2} \ln x}, \quad \int \frac{\text{d}x}{\sqrt{1 + x^2} \ln x}, \quad \int \frac{\text{d}x}{\sqrt{x^2 - 1} \ln x}.
    \]
\end{problem}\vspace{90pt}
\subsection{* 配凑法}
\begin{problem}
    计算以下不定积分:
    \[
        \int \dfrac{\text{d}x}{1 + x^3}, \quad \int \dfrac{\text{d}x}{1 + x^4}, \quad \int \dfrac{\text{d}x}{1 + x^5}.
    \]
\end{problem}\vspace{90pt}
\section{定积分的计算}
\subsection{定积分的定义与性质. Newton-Leibniz 公式}

\begin{problem}
    利用定积分定义计算:
    \[
        \lim_{n \to \infty} \sum_{k=1}^{n} \frac{k}{n^2}, \quad \lim_{n \to \infty} \sum_{k=1}^{n} \frac{1}{n + k}.
    \]
\end{problem}\vspace{90pt}

\begin{problem}
    计算极限:
    $$\lim_{n\to\infty}\left( \left( 1+\frac1n \right)\cdots\left( 1+\frac{n}{n} \right) \right)^{\frac{1}{n}}.$$
\end{problem}\vspace{90pt}
\subsection{换元法与分部积分法在定积分中的应用}
\begin{problem}
    计算 Fejer 积分:
    \[
        \int_0^{\pi} \frac{\sin^2(nx/2)}{\sin^2(x/2)} \, \text{d}x.
    \]
\end{problem}\vspace{90pt}
\begin{problem}
    计算以下定积分:
    \[
        \int_0^{\pi/2} x \sin x \, \text{d}x, \quad \int_0^1 \ln(1 + x^2) \, \text{d}x, \quad \int_0^{\pi/2} \ln(\sin x) \, \text{d}x.
    \]
\end{problem}\vspace{90pt}
\begin{problem}
    计算以下定积分:
    \[
        \int_0^1 \frac{\arctan x}{x} \, \text{d}x, \quad \int_0^{\pi/2} \frac{x}{\sin x} \, \text{d}x, \quad \int_0^{\pi/2} \frac{x}{\tan x} \, \text{d}x.
    \]
\end{problem}\vspace{90pt}
\begin{problem}
    计算以下定积分:
    \[
        \int_0^{\pi/2} \sin^n x \, \text{d}x, \quad \int_0^{\pi/2} \cos^n x \, \text{d}x.
    \]
\end{problem}\vspace{90pt}
\begin{problem}
    计算以下定积分:
    \[
        \int_0^{\pi} \frac{\text{d}x}{1 + \varepsilon \cos x} (|\varepsilon|<1).
    \]
\end{problem}\vspace{90pt}
\subsection{对称性与定积分的计算}
\begin{problem}
    计算以下定积分:
    \[
        \int_0^{\pi} \frac{x \sin x}{1 + \cos^2 x} \, \text{d}x, \quad \int_0^{\pi} \frac{x \sin x}{2 + \cos x} \, \text{d}x, etc.
    \]
\end{problem}\vspace{90pt}
\begin{problem}
    计算以下定积分:
    \[
        \int_0^{\pi/2} \ln(\sin x) \, \text{d}x, \quad \int_0^{\pi/2} \ln(\cos x) \, \text{d}x, \quad \int_0^{\pi/2} \ln(\tan x) \, \text{d}x.
    \]
\end{problem}\vspace{90pt}
\begin{problem}
    计算以下定积分:
    \[
        \int_0^{\pi/2} \frac{\text{d}x}{1 + \tan^n x}.
    \]
\end{problem}\vspace{90pt}
\begin{problem}
    设$f \in C[0,a], a>0.$ 若有 $f(x)f(a-x)\equiv 1,$ 计算下列定积分:
    $$\int_0^a \frac{\text{d}x}{f(x)+1}.$$
\end{problem}\vspace{90pt}
\subsection{反常积分的判敛}

\begin{problem}
    判定下列反常积分的敛散性:
    \[
        \quad \int_1^{\infty} \frac{1}{x^p} \, \text{d}x, \quad \int_0^1 \frac{1}{x^p} \, \text{d}x, \quad \int_0^{\infty} x^p\ln x \, \text{d}x.
    \]
\end{problem}\vspace{90pt}
\begin{problem}
    判定下列反常积分的敛散性:
    \[
        \quad \int_1^{\infty} \frac{\sin x}{x^p} \, \text{d}x, \quad \int_0^1 \frac{\sin x}{x^p} \, \text{d}x.
    \]
\end{problem}\vspace{90pt}
\begin{problem}
    判定下列反常积分的敛散性:
    \[
        \quad \int_0^{\infty} x^p{\sin x^q} \, \text{d}x.
    \]
\end{problem}\vspace{90pt}
\begin{problem}
    判定下列反常积分的敛散性:
    \[
        \quad \int_0^{\infty} \frac{\sin x}{x^p-\sin x} \, \text{d}x.
    \]
\end{problem}\vspace{90pt}
\begin{problem}
    判定下列反常积分的敛散性:
    \[
        \quad \int_0^{\infty} \frac{\text{d}x}{e^x - x^p}, \quad \int_0^{\infty} \frac{\text{d}x}{\ln(x)^p}.
    \]
\end{problem}\vspace{90pt}
\begin{problem}
    判定下列反常积分的敛散性:
    \[
        \int_{0}^{\infty} \sin(x)\sin(x^2) \text{d}x.
    \]
\end{problem}\vspace{90pt}

\begin{problem}
    对于连续函数$f \in C[a,+\infty)$, 且$\int_{a}^{+\infty} f(x) \, \text{d}x$存在,  证明:

    $$f\text{一致连续} \Leftrightarrow f(+\infty) = 0.$$
\end{problem}\vspace{90pt}

\subsection{反常积分的计算}
\begin{problem}
    计算以下反常积分:
    \[
        \int_0^{\infty} e^{-ax} \, \text{d}x, \quad \int_0^{\infty} x^n e^{-ax} \, \text{d}x, \quad (a>0, n>-1)
    \]
\end{problem}\vspace{90pt}
\begin{problem}
    利用 Fejer 积分, 计算 Dirichlet 积分:
    \[
        \int_0^{\infty} \frac{\sin x}{x} \, \text{d}x.
    \]
\end{problem}\vspace{90pt}
\begin{problem}
    计算下列 Gauss 积分:
    \[
        \int_{-\infty}^{\infty} e^{-x^2} \, \text{d}x, \quad \int_{-\infty}^{\infty} x^{2n} e^{-x^2} \, \text{d}x.
    \]
\end{problem}\vspace{90pt}
\begin{problem}
    计算以下反常积分:
    \[
        \int_0^{\infty} \frac{\text{d}x}{(1+x^2)^n}, \quad \int_0^{\infty} \frac{\text{d}x}{(x^2 + ax + b)^n} (4b > a^2).
    \]
\end{problem}\vspace{90pt}
\begin{problem}
    计算下列 Froullani 积分:
    \[
        \int_0^{\infty} \frac{f(ax) - f(bx)}{x} \, \text{d}x,
    \]
    其中 $a,b>0,$ 且 $f(+\infty)$ 存在.
\end{problem}\vspace{90pt}
\subsection{* Mobius 变换与有理函数的积分: 倒代换与等域变换. }
\begin{problem}
    计算以下反常积分:
    \[
        \int_0^\infty \frac{\text{d}x}{(x^2 + 1)(x^\alpha+1)}, \quad \int_0^\infty \frac{\ln(x+1)}{x^2 + 1} \, \text{d}x.
    \]
\end{problem}\vspace{90pt}
\section{积分不等式与积分中值定理}
\subsection{积分的估计}
\begin{problem}
    设 $f$ 在 $\left[ a,b \right]$ 上连续可导. 

(1) 证明: $$\max_{x \in [a,b]} |f(x) | \leq |\bar{f}| + \int_a^b |f'(x)| \, \text{d}x$$
    
(2)证明:
    \[
    \max_{x \in [a,b]} |f(x) - \bar{f}| \leq \frac{(b-a)}{2} \max_{x \in [a,b]} |f'(x)|,
    \]
    其中 $\bar{f} = \dfrac{1}{b-a} \int_a^b f(x) \, \text{d}x$.
\end{problem}\vspace{90pt}

\begin{problem}
    证明: 对于连续函数 $f(x) \in C[a,b]$ 与 $\xi \in (a,b)$, 有
    \[
        \lim_{h \to 0^+} \int_{a}^{b} f(x) \frac{h}{(x-\xi)^2+h^2} \, \text{d}x = \pi f(\xi).
    \]
\end{problem}\vspace{90pt}

\subsection{Cauchy 不等式. Young 不等式. Holder 不等式}

\begin{problem}[Young 不等式]
    设 $f(x), g(x) \in C[a,b]$ 且均为非负函数, 证明: 对于 $p,q>1$ 且 $\frac{1}{p} + \frac{1}{q} = 1,$ 有
    \[
        \int_a^b f(x) g(x) \, \text{d}x \leq \left( \int_a^b f^p(x) \, \text{d}x \right)^{\frac{1}{p}} \left( \int_a^b g^q(x) \, \text{d}x \right)^{\frac{1}{q}}.
    \]
\end{problem}\vspace{90pt}
\begin{problem}[Holder 不等式]
    设 $f(x), g(x) \in C[a,b]$ 且均为非负函数, 证明: 对于 $p,q>1$ 且 $\frac{1}{p} + \frac{1}{q} = 1,$ 有
    \[
        \int_a^b f(x) g(x) \, \text{d}x \leq \left( \int_a^b f^p(x) \, \text{d}x \right)^{\frac{1}{p}} \left( \int_a^b g^q(x) \, \text{d}x \right)^{\frac{1}{q}}.
    \]
\end{problem}\vspace{90pt}
\begin{problem}[Cauchy 不等式]
    设 $f(x), g(x) \in C[a,b]$, 证明:
    \[
        \left( \int_a^b f(x) g(x) \, \text{d}x \right)^2 \leq \left( \int_a^b f^2(x) \, \text{d}x \right) \left( \int_a^b g^2(x) \, \text{d}x \right).
    \]
\end{problem}\vspace{90pt}


\begin{problem}
    对于 满足 $f(0) = f(1) = 0$ 的 $C^1[0,1]$ 函数 $ f(x) $, 证明:
    \[
        \left( \int_0^1 xf(x) \, \text{d}x \right)^2 \leq \frac{1}{45} \int_0^1 (f'(x))^2 \, \text{d}x.
    \]
\end{problem}\vspace{90pt}




\subsection{Jensen 不等式}
\begin{problem}
    证明: 对于 $\int_a^b p(x) = 1,$ 对于任意 $f(x)>0 (x\in[a,b])$ 有
    $$\exp\left(\int_a^b p(x) \ln f(x) \, \text{d}x\right) \leq \int_a^b p(x) f(x) \, \text{d}x\leq \ln(\int_a^b p(x) \exp(f(x)) \, \text{d}x).$$
\end{problem}\vspace{90pt}
\begin{problem}
    证明: 对于下凸函数 $f(x)$, 有
    \[
        f\left(\int_a^b p(x)  \, \text{d}x\right) \leq \int_a^b f(p(x)) \, \text{d}x.
    \]
    特别的, 取 $p(x) = x^n,$ 得到什么结果?
\end{problem}\vspace{90pt}
\subsection{中值定理与 Taylor 公式. *Darboux 公式}
\begin{problem}
    设 $f(x)$ 在 $[a,b]$ 上具有 $n+1$ 阶导数, 证明: 存在 $\xi \in (a,b)$ 使得
    \[
        \int_a^b f(x) \, \text{d}x = \sum_{k=0}^{n} \frac{f^{(k)}(a)}{(k+1)!} (b-a)^{k+1} + \frac{f^{(n+1)}(\xi)}{(n+2)!} (b-a)^{n+2}.
    \]
\end{problem}\vspace{90pt}

\begin{problem}
    设 $f$ 在 $[a,b]$ 上二阶连续可导, 证明:
    \[
        \int_a^b f(x) \, \text{d}x = (b-a) f\left( \frac{a+b}{2} \right) + \frac{(b-a)^3}{24} f''(\xi),
    \]
    其中 $\xi \in (a,b)$.
\end{problem}\vspace{90pt}

\begin{problem}
    设 $f(x)$ 在 $[a,b]$ 上具有连续的 $n$ 阶导数, 证明: Taylor 公式的积分余项形式为
    \[
        f(b) = \sum_{k=0}^{n-1} \frac{f^{(k)}(a)}{k!} (b-a)^k + \int_a^b \frac{f^{(n)}(t)}{(n-1)!} (b-t)^{n-1} \, dt.
    \]
\end{problem}\vspace{90pt}

\begin{problem}[*Darboux 公式]
    设 $f(x)$ 在 $[a,b]$ 上具有 $n+1$ 阶导数, $\phi(t)$ 是 $n$ 阶多项式. 证明: 
    $$
        \phi^{(n)}(0) (f(b)-f(a)) = \sum_{k=1}^{n} (-1)^{k-1} \left( \phi^{(n-k)}(1) f^{(k)}(b) - \phi^{(n-k)}(0) f^{(k)}(a) \right) + (-1)^n \int_0^1 \phi(t) f^{(n+1)}(a + t(b-a)) (b-a)^{n+1} \, dt.
    $$
\end{problem}\vspace{90pt}
\subsection{* Taylor 展开与积分的估计}
\begin{problem}
    对于函数 $f(x) = e^{x^2}(\int_{x}^{+\infty} e^{-t^2} \, dt)$, 证明: 在$x \to +\infty$ 时, 成立展开
    \[
        f(x) = \frac{1}{2x} - \frac{1}{4x^3} + \frac{3}{8x^5} - \cdots + (-1)^{n} \frac{(2n-1)!!}{2^{n+1} x^{2n+1}} + R_n(x),
    \]
    其中余项 $R_n(x)$ 满足
    \[
        |R_n(x)| \leq \frac{(2n+1)!!}{2^{n+1} x^{2n+3}}.
    \]
\end{problem}\vspace{90pt}
\section{常微分方程与一元微积分学在物理学中的应用}
\subsection{可分离变量的微分方程}
这一部分是关于 $y' = f(y)g(x)$ 的微分方程. 
\begin{problem}
    求解下列微分方程:
    \[
        y' = y^2 \sin x, \quad y' = \frac{y}{x} + \frac{y^2}{\sqrt{x}}, \quad (x>0).
    \]
\end{problem}\vspace{90pt}
\subsection{一阶线性微分方程}
\begin{problem}
    求解下列微分方程:
    \[
        y' + y \tan x = \sin x, \quad y' - \frac{2}{x} y = x^2, \quad (x \neq 0).
    \]
\end{problem}\vspace{90pt}
\begin{problem}
    求解下列微分方程:
    \[
        y' + \frac{1}{x} y = \frac{\ln x}{x}, \quad y' + 2xy = x e^{-x^2}.
    \]
\end{problem}\vspace{90pt}

\subsection{二阶线性微分方程. 常数变易法}
\begin{problem}
    求解下列微分方程:
    \[
        y'' - y = e^x, \quad y'' + 4y = \sin 2x.
    \]
\end{problem}\vspace{90pt}
\begin{problem}
    求解下列微分方程:
    \[
        y'' + y' - 2y = e^{3x}, \quad y'' - 2y' + y = e^x.
    \]
\end{problem}\vspace{90pt}
\subsection{可以降阶的微分方程}
\begin{problem}
    求解下列微分方程:
    \[
        y'' = (y')^2, \quad y'' = y' \ln y'.
    \]
\end{problem}\vspace{90pt}
\begin{problem}[Euler 方程]
    求解下列微分方程:
    \[
        x^2 y'' - 3x y' + 4y = x^2, \quad x^2 y'' + x y' - y = \ln x.
    \]
\end{problem}\vspace{90pt}
\begin{problem}[Bernoulli 方程]
    求解下列微分方程:
    \[
        y'' + y = \sec^2 x \cdot y^2, \quad y'' - y = e^x y^3.
    \]
\end{problem}\vspace{90pt}
\subsection{微分方程在物理学中的应用}
\begin{problem}
    设一质量为 $m$ 的物体在重力作用下自由下落, 且受到与速度成正比的阻力. 试建立物体运动的微分方程, 并求解该微分方程, 给出物体的速度 $v$ 随时间 $t$ 的变化关系.
\end{problem}\vspace{90pt}

\begin{problem}[Tsiolkovsky 公式]
    试推导火箭运动的 Tsiolkovsky 公式, 也就是:
    \[
        v_f - v_i = v_e \ln \frac{m_i}{m_f},
    \]
    其中 $v_f$ 和 $v_i$ 分别是火箭在燃料燃尽时和起飞时的速度, $v_e$ 是燃料相对于火箭的排出速度, $m_i$ 和 $m_f$ 分别是火箭起飞时和燃料燃尽时的质量.
\end{problem}\vspace{90pt}

\begin{problem}[相对论性动力学]

    (1) 设一质量为 $m_0$ 的粒子在受力 $F$ 作用下运动, 其动量 $p$ 与速度 $v$ 的关系为 $p = \frac{m_0 v}{\sqrt{1 - \frac{v^2}{c^2}}}$, 其中 $c$ 是光速. 试建立粒子运动的微分方程, 并求解该微分方程, 给出粒子的速度 $v$ 随时间 $t$ 的变化关系.
    
    (2) 引入本征时间 $\tau$, 使得 $d\tau = dt \sqrt{1 - \frac{v^2}{c^2}}$. 试用本征时间重新推导粒子的运动方程, 并求解该方程, 给出粒子的速度 $v$ 随本征时间 $\tau$ 的变化关系.
\end{problem}\vspace{90pt}
\subsection{积分学在物理学中的应用}

\begin{problem}
    设一均匀细杆长为 $L$, 线密度为 $\lambda$, 求杆在距其一端 $d$ 处的引力.
\end{problem}\vspace{90pt}

\begin{problem}[浮力]
    将 半径为 $R$ 的球体完全浸入液体中,求克服浮力做的功。
\end{problem}\vspace{90pt}

\begin{problem}[圆环近轴引力势能]
    对于质量密度为 $\lambda$ 的圆环,半径为 $R$,求其在轴线上距离圆环中心 $d$ 处的引力势能。 展开到$O(d^2)$阶.
\end{problem}
\end{document}
