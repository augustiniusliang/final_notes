\documentclass[UTF8]{ctexart}
\usepackage{amsmath, amssymb, amsthm}
\usepackage{geometry}
\geometry{a4paper, scale=0.8}
\usepackage{graphicx}
\title{高等微积分期末辅导}
\author{梁莫言, 清华大学未央书院.}
\newtheorem{problem}{问题}
\date{2025 年 1 月 5 日}

\begin{document}
\maketitle
\tableofcontents
\newpage

\section{不定积分的计算}
\subsection{分部积分法}
\begin{problem}
  计算以下不定积分:
  \[
    \int x^2 e^x \, dx, \quad \int \ln x \, dx, \quad \int e^x \sin x \, dx.
  \]
\end{problem}
\begin{problem}
    计算以下不定积分:
    \[
        \int \cos^5 x \,dx, \quad \int \arcsin x \, dx, \quad \int \sec^3 x \, dx.
    \]
\end{problem}
\begin{problem}
    推到下列积分的递推公式:
    \[
        I_n = \int \sin^n x \, dx, \quad J_n = \int x^n e^x \, dx, \quad K_n = \int \tan^n x \, dx.
    \]
\end{problem}

\subsection{换元法. 三角换元, 双曲三角换元}
\begin{problem}
    计算以下不定积分:
    \[
        \int \frac{dx}{x^2 \sqrt{x^2 - 4}}, \quad \int \frac{dx}{\sqrt{1 - x^2}}, \quad \int \frac{dx}{\sqrt{x^2 + 4}}.
    \]
\end{problem}
\begin{problem}
    计算以下不定积分:
    \[
        \int \frac{dx}{x \sqrt{x^2 - x + 1}}, \quad \int \frac{dx}{\sqrt{x^2 + x + 1}}, \quad \int \frac{dx}{x^2 \sqrt{4 - x^2}}.
    \]
\end{problem}
\begin{problem}
    计算以下不定积分:
    \[
        \int \sinh^3 x \, dx, \quad \int \frac{dx}{\cosh x}, \quad \int \frac{dx}{\sinh x}.
    \]
\end{problem}
\begin{problem}
    计算以下不定积分:
    \[
        \int \frac{dx}{\sqrt{x^2 + 1}}, \quad \int \frac{dx}{\sqrt{x^2 - 1}}, \quad \int \frac{dx}{\sqrt{1 - x^2}}.
    \]
\end{problem}
\subsection{有理分式的化简与原函数}
\begin{problem}
    计算以下不定积分:
    \[
        \int \frac{2x^2 + 3x + 1}{(x - 1)(x^2 + 1)} \, dx, \quad \int \frac{x^3 + 1}{x^2 - x} \, dx.
    \]
\end{problem}
\begin{problem}
    计算以下不定积分:
    \[
        \int \frac{dx}{x^3 - x^2}, \quad \int \frac{dx}{x^4 + 4}, \quad \int \frac{dx}{x^4 + x^2 + 1}.
    \]
\end{problem}
\subsection{超越函数的原函数}
\subsection{* 配凑法}
\begin{problem}
    计算以下不定积分:
    \[
        \int \dfrac{dx}{1 + x^3}, \quad \int \dfrac{dx}{1 + x^4}, \quad \int \dfrac{dx}{1 + x^5}.
    \]
\end{problem}
\section{定积分的计算}
\subsection{定积分的定义与性质. Newton-Leibniz 公式}

\begin{problem}
    利用定积分定义计算:
    \[
        \lim_{n \to \infty} \sum_{k=1}^{n} \frac{k}{n^2}, \quad \lim_{n \to \infty} \sum_{k=1}^{n} \frac{1}{n + k}.
    \]
\end{problem}

\begin{problem}
    计算极限:
    $$\lim_{n\to\infty}\left( \left( 1+\frac1n \right)\cdots\left( 1+\frac{n}{n} \right) \right)^{\frac{1}{n}}.$$
\end{problem}
\subsection{换元法与分部积分法在定积分中的应用}
\begin{problem}
    计算 Fejer 积分:
    \[
        \int_0^{\pi} \frac{\sin^2(nx/2)}{\sin^2(x/2)} \, dx.
    \]
\end{problem}
\begin{problem}
    计算以下定积分:
    \[
        \int_0^{\pi/2} x \sin x \, dx, \quad \int_0^1 \ln(1 + x^2) \, dx, \quad \int_0^{\pi/2} \ln(\sin x) \, dx.
    \]
\end{problem}
\begin{problem}
    计算以下定积分:
    \[
        \int_0^1 \frac{\arctan x}{x} \, dx, \quad \int_0^{\pi/2} \frac{x}{\sin x} \, dx, \quad \int_0^{\pi/2} \frac{x}{\tan x} \, dx.
    \]
\end{problem}
\begin{problem}
    计算以下定积分:
    \[
        \int_0^{\pi/2} \sin^n x \, dx, \quad \int_0^{\pi/2} \cos^n x \, dx.
    \]
\end{problem}
\begin{problem}
    计算以下定积分:
    \[
        \int_0^{\pi} \frac{dx}{1 + \varepsilon \cos x} (|\varepsilon|<1).
    \]
\end{problem}
\subsection{对称性与定积分的计算}
\begin{problem}
    计算以下定积分:
    \[
        \int_0^{\pi} \frac{x \sin x}{1 + \cos^2 x} \, dx, \quad \int_0^{\pi} \frac{x \sin x}{2 + \cos x} \, dx, etc.
    \]
\end{problem}
\begin{problem}
    计算以下定积分:
    \[
        \int_0^{\pi/2} \ln(\sin x) \, dx, \quad \int_0^{\pi/2} \ln(\cos x) \, dx, \quad \int_0^{\pi/2} \ln(\tan x) \, dx.
    \]
\end{problem}
\begin{problem}
    计算以下定积分:
    \[
        \int_0^{\pi/2} \frac{dx}{1 + \tan^n x}.
    \]
\end{problem}
\begin{problem}
    设$f \in C[0,a], a>0.$ 若有 $f(x)f(a-x)\equiv 1,$ 计算下列定积分:
    $$\int_0^a \frac{dx}{f(x)+1}.$$
\end{problem}
\subsection{反常积分的判敛}

\begin{problem}
    判定下列反常积分的敛散性:
    \[
        \quad \int_1^{\infty} \frac{1}{x^p} \, dx, \quad \int_0^1 \frac{1}{x^p} \, dx, \quad \int_0^{\infty} x^p\ln x \, dx.
    \]
\end{problem}
\begin{problem}
    判定下列反常积分的敛散性:
    \[
        \quad \int_1^{\infty} \frac{\sin x}{x^p} \, dx, \quad \int_0^1 \frac{\sin x}{x^p} \, dx.
    \]
\end{problem}
\begin{problem}
    判定下列反常积分的敛散性:
    \[
        \quad \int_0^{\infty} x^p{\sin x^q} \, dx.
    \]
\end{problem}
\begin{problem}
    判定下列反常积分的敛散性:
    \[
        \quad \int_0^{\infty} \frac{\sin x}{x^p-\sin x} \, dx.
    \]
\end{problem}
\begin{problem}
    判定下列反常积分的敛散性:
    \[
        \quad \int_0^{\infty} \frac{dx}{e^x - x^p}, \quad \int_0^{\infty} \frac{dx}{\ln(x)^p}.
    \]
\end{problem}
\begin{problem}
    判定下列反常积分的敛散性:
    \[
        \int_{0}^{\infty} \sin(x)\sin(x^2) dx.
    \]
\end{problem}

\subsection{反常积分的计算}
\begin{problem}
    计算以下反常积分:
    \[
        \int_0^{\infty} e^{-ax} \, dx, \quad \int_0^{\infty} x^n e^{-ax} \, dx, \quad (a>0, n>-1)
    \]
\end{problem}
\begin{problem}
    计算下列 Gauss 积分:
    \[
        \int_{-\infty}^{\infty} e^{-x^2} \, dx, \quad \int_{-\infty}^{\infty} x^{2n} e^{-x^2} \, dx.
    \]
\end{problem}
\subsection{* Mobius 变换与有理函数的积分: 倒代换与等域变换. }
\begin{problem}
    计算以下反常积分:
    \[
        \int_0^\infty \frac{dx}{(x^2 + 1)(x^\alpha+1)}, \quad \int_0^\infty \frac{\ln(x+1)}{x^2 + 1} \, dx.
    \]
\end{problem}
\section{积分不等式与积分中值定理}
\subsection{积分的估计}
\begin{problem}
    设 $f$ 在 $\left[ a,b \right]$ 上连续可导. 

(1) 证明: $$\max_{x \in [a,b]} |f(x) | \leq |\bar{f}| + \int_a^b |f'(x)| \, dx$$
    
(2)证明:
    \[
    \max_{x \in [a,b]} |f(x) - \bar{f}| \leq \frac{(b-a)}{2} \max_{x \in [a,b]} |f'(x)|,
    \]
    其中 $\bar{f} = \dfrac{1}{b-a} \int_a^b f(x) \, dx$.
\end{problem}

\begin{problem}
    证明: 对于连续函数 $f(x) \in C[a,b]$ 与 $\xi \in (a,b)$, 有
    \[
        \lim_{h \to 0^+} \int_{a}^{b} f(x) \frac{h}{(x-\xi)^2+h^2} \, dx = \pi f(\xi).
    \]
\end{problem}

\subsection{中值定理与积分不等式}
\subsection{Cauchy 不等式. Young 不等式. Holder 不等式}

\subsection{Jensen 不等式}
\begin{problem}
    证明: 对于 $\int_a^b p(x) = 1,$ 对于任意 $f(x)>0 (x\in[a,b])$ 有
    $$\exp\left(\int_a^b p(x) \ln f(x) \, dx\right) \leq \int_a^b p(x) f(x) \, dx\leq \ln(\int_a^b p(x) \exp(f(x)) \, dx).$$
\end{problem}
\subsection{中值定理与 Taylor 公式. *Darboux 公式}
\begin{problem}
    设 $f(x)$ 在 $[a,b]$ 上具有 $n+1$ 阶导数, 证明: 存在 $\xi \in (a,b)$ 使得
    \[
        \int_a^b f(x) \, dx = \sum_{k=0}^{n} \frac{f^{(k)}(a)}{(k+1)!} (b-a)^{k+1} + \frac{f^{(n+1)}(\xi)}{(n+2)!} (b-a)^{n+2}.
    \]
\end{problem}

\begin{problem}
    设 $f(x)$ 在 $[a,b]$ 上具有连续的 $n$ 阶导数, 证明: Taylor 公式的积分余项形式为
    \[
        f(b) = \sum_{k=0}^{n-1} \frac{f^{(k)}(a)}{k!} (b-a)^k + \int_a^b \frac{f^{(n)}(t)}{(n-1)!} (b-t)^{n-1} \, dt.
    \]
\end{problem}

\begin{problem}[*Darboux 公式]
    设 $f(x)$ 在 $[a,b]$ 上具有 $n+1$ 阶导数, $\phi(t)$ 是 $n$ 阶多项式. 证明: 
    $$
        \phi^{(n)}(0) (f(b)-f(a)) = \sum_{k=1}^{n} (-1)^{k-1} \left( \phi^{(n-k)}(1) f^{(k)}(b) - \phi^{(n-k)}(0) f^{(k)}(a) \right) + (-1)^n \int_0^1 \phi(t) f^{(n+1)}(a + t(b-a)) (b-a)^{n+1} \, dt.
    $$
\end{problem}
\section{常微分方程与一元微积分学在物理学中的应用}
\subsection{可分离变量的微分方程}
\subsection{一阶线性微分方程}
\subsection{二阶线性微分方程. 常数变易法}
\subsection{可以降阶的微分方程}
\subsection{微分方程在物理学中的应用}
\subsection{积分学在物理学中的应用}
\end{document}
